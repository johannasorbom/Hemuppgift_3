\documentclass{report}
\usepackage[utf8]{inputenc}
\usepackage[swedish]{babel}
\usepackage{url}
\usepackage{hyperref}
\title{}
\date{\today}
\author{Johanna Sörbom}

\newcommand{\case}[1]{\subsection*{#1}}
\newcommand{\ac}{AC-system}
\newcommand{\cmp}[2]{\ensuremath{#1+#2i}}
\begin{document}
\maketitle
\section{Inledning}
\section{Metod}
\section{Bakgrund}
\section{Resultat och beräkningar}
\subsection{Impedance of power lines}
The impedence of the power line is given by the equation 

\begin{equation}\label{eq_impd}
Z =  L \cdot Z_{km}
\end{equation} where $L$ is the length of the power line in kilometers and $Z_{km}$ is the impedience per kilometer in Ohms.

\case{Ställverket to Brytpunkten}
For the case of Ställverket to Brytpunkten, we have $L=10km$ and $Z_{km}=\cmp{0.2}{0.4}$.  We get,

\begin{eqnarray}
Z&=&  L \cdot Z_{km} \\
&=&10 (\cmp{0.2}{0.4}) \\
&=& \cmp{2}{4} \label{res}
\end{eqnarray} by substituing into equation \ref{eq_impd}. Interpreting the real part of the result obtained in equation \ref{res} as resistance and the complex part as reactance, we have $R=2\Omega$ and $X=4\Omega$ respectively.

\case {Brytpunkten to Framtiden}
For the case of Brytpunkten to Framtiden, we have $L=15km$ and $Z_{km}=\cmp{0.2}{0.4}$.  We get,

\begin{eqnarray}
Z&=&  L \cdot Z_{km} \\
&=&15 (\cmp{0.2}{0.4}) \\
&=& \cmp{3}{6} \label{res}
\end{eqnarray} by substituing into equation \ref{eq_impd}. Interpreting the real part of the result obtained in equation \ref{res} as resistance and the complex part as reactance, we have $R=3\Omega$ and $X=6\Omega$ respectively.

\case {Brytpunkten to Solsidan}
For the case of Brytpunkten to Solsidan, we have $L=21.21km$ and $Z_{km}=\cmp{0.2}{0.4}$.  We get,

\begin{eqnarray}
Z&=&  L \cdot Z_{km} \\
&=&21.21 (\cmp{0.2}{0.4}) \\
&=& \cmp{4.24}{8.49} \label{res}
\end{eqnarray} by substituing into equation \ref{eq_impd}. Interpreting the real part of the result obtained in equation \ref{res} as resistance and the complex part as reactance, we have $R=3\Omega$ and $X=6\Omega$ respectively.

\case {Framtiden to Solsidan}
For the case of Framtiden to Solsidan, we have $L=15km$ and $Z_{km}=\cmp{0.3}{0.1}$.  We get,

\begin{eqnarray}
Z&=&  L \cdot Z_{km} \\
&=&15 (\cmp{0.3}{0.1}) \\
&=& \cmp{4.5}{1.5} \label{res}
\end{eqnarray} by substituing into equation \ref{eq_impd}. Interpreting the real part of the result obtained in equation \ref{res} as resistance and the complex part as reactance, we have $R=4.5\Omega$ and $X=1.5\Omega$ respectively.

\case {Vindeby to Solsidan}
For the case of Vindeby to Solsidan, we have $L=15km$ and $Z_{km}=\cmp{0.3}{0.1}$.  We get,

\begin{eqnarray}
Z&=&  L \cdot Z_{km} \\
&=&15 (\cmp{0.2}{0.4}) \\
&=& \cmp{3}{6} \label{res}
\end{eqnarray} by substituing into equation \ref{eq_impd}. Interpreting the real part of the result obtained in equation \ref{res} as resistance and the complex part as reactance, we have $R=3\Omega$ and $X=6\Omega$ respectively.

\subsection{Calculation of reactive power}
Complex power is defined as, \begin{equation}
S = \sqrt{P^2 - Q^2} \label{s}
\end{equation}
where $P$ is the active power and $Q$ is the reactive power. For balanced load in an \ac we have, 
\begin{equation}
P = S \cdot \cos\theta
\end{equation}
where $\theta$ is the phase offset between voltage and current. 

\case{Brytpunkten}
For the case of Brytpunkten where $\cos\theta = 0.9$ and $P = 35 MW$ we get,
 
\begin{eqnarray}
Q&=&   = \sqrt{S^2 - P^2}\\
&=& \sqrt{38.89^2 - 35^2} \\
&=& 16.95 MW \label{res}
\end{eqnarray} 
by substituting into equation \ref{s}.

\case{Framtiden}
For the case of Framtiden where $\cos\theta = 0.9$ and $P = 10 MW$ we get,
 
\begin{eqnarray}
Q&=&   = \sqrt{S^2 - P^2}\\
&=& \sqrt{11.11^2 - 10^2} \\
&=& 4.84 MW \label{res}
\end{eqnarray} 
by substituting into equation \ref{s}.

\case{Solsidan}
For the case of Solsidan where $\cos\theta = 0.9$ and $P = 5 MW$ we get,
 
\begin{eqnarray}
Q&=&   = \sqrt{S^2 - P^2}\\
&=& \sqrt{5.56^2 - 5^2} \\
&=& 2.43 MW \label{res}
\end{eqnarray} by substituting into equation \ref{s}.



\end{document}
