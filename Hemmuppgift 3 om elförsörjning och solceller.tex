\documentclass{report}
\usepackage[utf8]{inputenc}
\usepackage[swedish]{babel}
\usepackage{url}
\usepackage{hyperref}
\title{}
\date{\today}
\author{Johanna Sörbom}

\newcommand{\case}[1]{\subsection*{#1}}
\newcommand{\cmp}[2]{\ensuremath{#1+#2i}}
\begin{document}
\maketitle
\section{Inledning}
\section{Metod}
\section{Bakgrund}
\section{Resultat och beräkningar}
The impedence of the power line is given by the equation 

\begin{equation}\label{eq_impd}
Z =  L \cdot Z_{km}
\end{equation} where $L$ is the length of the power line in kilometers and $Z_{km}$ is the impedience per kilometer in Ohms.

\case{Ställverket to Brytpunkten}
For the case of X to Y, we have $L=10km$ and $z=\cmp{0.2}{0.4}$.  We get,

\begin{eqnarray}
Z&=&  L \cdot Z_{km} \\
&=&10 (\cmp{0.2}{0.4}) \\
&=& \cmp{2}{4} \label{res}
\end{eqnarray} by substituing into equation \ref{eq_impd}. Interpreting the real part of the result obtained in equation \ref{res} as resistance and the complex part as reactance, we have $R=2$ and $X=4$ respectively.



\end{document}
